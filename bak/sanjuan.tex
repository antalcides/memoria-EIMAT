\documentclass[12pt]{book}
\usepackage{graphicx}
\usepackage[spanish]{babel}
\usepackage{fancyhdr}               %Paquete para dar forma al encabezado y pie de p\'{a}gina
\usepackage[spanish]{babel}
\usepackage[latin1]{inputenc}
\pagestyle{fancy}\fancyhf{}\fancyhead[C]{\textit{An�lisis Funcional No-lineal}}     
\fancyfoot[R]{\scriptsize \emph{\thepage}}\fancyfoot[L]{\scriptsize \emph{Universidad
 del Atl�ntico Programa de Matem�ticas.  Barranquilla, Colombia}}
\renewcommand{\headrulewidth}{0.6pt} % Ancho de la l\'{\i}nea horizontal bajo el encabezado
\renewcommand{\footrulewidth}{0.6pt} % Ancho de la l\'{\i}nea horizontal sobre el pie (que en
%este ejemplo est\'{a} vac\'{\i}o)
\setlength{\headheight}{1.5\headheight} % Aumenta la altura del encabezado en una vez y media
\pagestyle{fancy}
\begin{document}

\noindent
\begin{minipage}{400pt}
\vspace*{-4.7cm}
%\begin{flushright}
\textbf{X Encuentro Internacional de Matem�ticas EIMAT 2014}\\
%\end{flushright}
\end{minipage}
\\ [.5cm]



\thispagestyle{fancy}
\begin{center}
 \large
 \textbf{Bifurcaci�n de Soluciones al Problema de la Cuerda Vibrante}
\end{center}
\vspace{0.5cm}

\begin{center}
\textbf{ARTURO SANJU�N}
\end{center}
    \begin{center}
    UNIVERSIDAD DISTRITAL FRANCISCO JOS� DE CALDAS
    \emph{aasanjuanc@udistrital.edu.co}
    \end{center}



\vspace{1.5cm}
\section*{Resumen}

Presentamos aplicaciones de la teor�a de Bifurcaciones como el Teorema de
Krasonosel'skii-Rabinowitz \cite{Bro04} y otros \cite{Dem85} a la ecuaci�n de onda semilineal.

La bifurcaci�n en infinito de la ecuaci�n de onda no-lineal est� poco documentada y se presentar�n
algunos ejemplos al respecto.

Esta ponencia est� enmarcada en la investigaci�n doctoral del autor dirigida por los
profesores Francisco Caicedo y Alfonso Castro.

\begin{thebibliography}{99}

\bibitem{Bro04} R.F. Brown. \newblock {\em {A Topological Introduction to Nonlinear Analysis}}.
\newblock Birkh{\"a}user Boston, 2004.

\bibitem{CaCaiDu11}
J.~F. Caicedo, A.~Castro, and R.~Duque.
\newblock {Existence of Solutions for a wave equation with non-monotone
  nonlinearity}.
\newblock {\em Milan J. Math}, 79(1):207--222, 2011.

\bibitem{Dem85}
K.~Deimling.
\newblock {\em {Nonlinear functional analysis}}.
\newblock Springer-Verlag, 1985.

\bibitem{Rabi71}
P. Rabinowitz.
\newblock {Some global results for nonlinear eigenvalue problems}.
\newblock {\em Journal of Functional Analysis}, 7(3):487--513, 1971.

\end{thebibliography}


\end{document}