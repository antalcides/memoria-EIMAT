 \makeatletter
%%%%%%%%%%%%%%%%%%%%%%%%
%original de ldesc2e.sty 
\newcommand{\manual}{Manual de \emph{\LaTeX{}}~\cite{manual}} 
\newcommand{\companion}{\emph{The \LaTeX{} Companion}~\cite{companion}} 
\newcommand{\guia}{\emph{Gu\'{\i}a Local}~\cite{local}}
\newcommand{\contrib}[3]{#1\quad$<$\texttt{#2}$>$%
{\small\\\quad\textit{#3}}\\[1ex]}
%
% Algunas instrucciones para ayudar a la creaci'on del 'indice de
% materias.
%
%\newcommand{\bs}{\symbol{'134}}%Print backslash
\ifx\bs\undefined
  \newcommand{\bs}{\symbol{92}}%Print backslash
\else
  \renewcommand{\bs}{\symbol{92}}%Print backslash
\fi
%\newcommand{\bs}{\ensuremath{\mathtt{\backslash}}}%Imprime barra invertida
% Entrada en el 'indice para una orden
%
% Entrada de s'imbolo para la tabla de s'imbolos matem'aticos
%
\newcommand{\X}[1]{$#1$&\texttt{\string#1}\hspace*{1ex}}
% Text normal.... 
\newcommand{\SC}[1]{#1&\texttt{\string#1}\hspace*{1ex}}
% para los acentos en modo texto
\newcommand{\A}[1]{#1&\texttt{\string#1}\hspace*{1ex}}
\newcommand{\B}[2]{#1#2&\texttt{\string#1{} #2}\hspace*{1ex}}

\newcommand{\W}[2]{$#1{#2}$&
  \texttt{\string#1}\texttt{\string{\string#2\string}}\hspace*{1ex}}
\newcommand{\Y}[1]{$\big#1$ &\texttt{\string#1}}  %
% Tabla de s'imbolos matem'aticos
\newsavebox{\symbbox}
\newenvironment{symbols}[1]%
{\par\vspace*{2ex}
\renewcommand{\arraystretch}{1.1}
\begin{lrbox}{\symbbox}
\hspace*{4ex}\begin{tabular}{@{}#1@{}}}%
{\end{tabular}\end{lrbox}\makebox[\textwidth]{\usebox{\symbbox}}\par\medskip}
%
% Preparaci'on especial para imprimir los s'imblos de la AMS
% Si no se encuentra AMS, deber'ia funcionar.
%

% No se tienen versiones PS de los tipos rsfs.
% Por ello, esto no no se puede hacer para pdf
\ifx\pdfoutput\undefined % No estamos corriendo pdftex
\IfFileExists{mathrsfs.sty}
  {\RequirePackage{mathrsfs}\let\MathRSFS\mathscr\let\mathscr\relax}{}
\fi
\IfFileExists{amssymb.sty}
  {\let\noAMS\relax \RequirePackage{amssymb}}
  {\def\noAMS{\endinput}\RequirePackage{latexsym}}

\IfFileExists{euscript.sty}
  {\RequirePackage{euscript}}{}
%\IfFileExists{eufrak.sty}
%  {\RequirePackage{eufrak}}{}


%
% Imprimir |--| para mostrar distancia
%
\newcommand{\demowidth}[1]{\rule{0.3pt}{1.3ex}\rule{#1}{0.3pt}\rule{0.3pt}{1.3ex}}
\renewcommand{\cleardoublepage}
    {\clearpage\if@twoside \ifodd\c@page\else
    \hbox{}\thispagestyle{empty}\newpage\if@twocolumn\hbox{}\newpage\fi\fi\fi}

\newcommand{\BibTeX}
     {\textsc{Bib}\TeX}
%%%%% hasta aqui original de ldesc2e.sty
%%%%%%%%%%%%%%%%%%%%%%%%%%%%%%%%%%%%%%%%%%%%%%
\xcolortheorem[background=Tan!5 ,titlebackground=thered!40!black ,titleboxcolor=mybrown!40!black ,boxcolor=mybrown!40!black]{out1}{Salida}
\DeclareCaptionFormat{ejer}{\colorbox{thered!40!black}{\parbox{\dimexpr\textwidth-2\fboxsep\relax}{\color{white}Ejercicio \thesection .\ 
\theejercicio\hfill #3}}}
\captionsetup[ejer]{format=ejer,labelfont=white,textfont=white, singlelinecheck=false, margin=0pt, font={bf,footnotesize}}
\newcounter{ejercicio}[section]
\newwrite\ejercicio@out
\newenvironment{ejercicio}%
 {\begingroup% Lets Keep the Changes Local
  \@bsphack
  \immediate\openout \ejercicio@out \jobname.exa
  \let\do\@makeother\dospecials\catcode`\^^M\active
  \def\verbatim@processline{%
    \immediate\write\ejercicio@out{\the\verbatim@line}}%
  \verbatim@start}%
 {\immediate\closeout\ejercicio@out\@esphack\endgroup%
%
% Y aqu'i lo que se ha a~nadido

 \setlength{\parindent}{0pt}
    %\setlength{\parskip}{1ex plus 0.5ex minus 0.7ex}
     \noindent
  %  \hspace*{+2ex}
%  \setlength{\parindent}{0pt}
    \setlength{\parskip}{1ex plus 0.3ex minus 0.7ex}
            \makebox[0.45\linewidth][t]{
            \hspace{+2ex}
            \setlength{\parindent}{0pt}
              \begin{minipage}[t]{0.45\linewidth}
              \vspace{-3.5ex}
      \refstepcounter{ejercicio}
   \captionsetup{options=ejer}
   \lstset{%
   caption=Entrada,basicstyle=\tiny\ttfamily\bf,language={[LaTeX]TeX}, numbersep=5mm, numbers=left, numberstyle=\tiny, % number style
   breaklines=true,framexleftmargin=10mm, xleftmargin=10mm,
   backgroundcolor=\color{Tan!5},frameround=fttt,escapeinside=??,
   rulecolor=\color{thered!40!black},
   morekeywords={% Give key words here                                         % keywords
       maketitle},
   keywordstyle=\color[rgb]{0,0,1},                    % keywords
           commentstyle=\color[rgb]{0.133,0.545,0.133},    % comments
           stringstyle=\color[rgb]{0.627,0.126,0.941}  % strings
   %columns=fullflexible   
   }
    \lstinputlisting[]{\jobname.exa}
         \end{minipage}}%
      \hspace{30pt} 
           \hfill
  %
   \setlength{\parindent}{0pt}
   \setlength{\parskip}{1ex plus 0.5ex minus 0.7ex}
%   
  \makebox[0.5\linewidth][t]{%
%  %\colorbox{ptcbackground!60}{
  \setlength{\parindent}{0pt}
    \setlength{\parskip}{1ex plus 0.3ex minus 0.7ex}
      \begin{minipage}[t]{0.40\linewidth}
         \setlength{\parindent}{0pt}
  \setlength{\parskip}{1ex plus 0.4ex minus 0.7ex}
  \vspace*{-3.5ex} 
        \begin{trivlist}
     \scriptsize\bf\ttfamily \item\begin{out1}{Pdf}\input{\jobname.exa}\end{out1}
    \end{trivlist}
      \end{minipage}%}
      }
    
      \par\addvspace{3ex plus 1ex}\vskip -\parskip
} 
%%%%%%%%%%%%%%%%%%%%%%%%%%%%%%%%%%%%%%%%%%%%%%%%%%%%
\DeclareCaptionFont{white}{\color{white}}
\newcounter{example}[section]
\DeclareCaptionFormat{out}{\colorbox{mybrown!40!black}{\parbox{\dimexpr\textwidth-2\fboxsep\relax}{\color{white}C\'odigo del ejemplo\, \thesection .\ 
\theexample\hfill #3}}}
\captionsetup[out]{format=out,labelfont=white,textfont=white, singlelinecheck=false, margin=0pt, font={bf,footnotesize}}
\xcolortheorem[background=mybrown!5 ,titlebackground=mybrown!40!black ,titleboxcolor=mybrown!40!black ,boxcolor=mybrown!40!black]{out}{Salida}
\newwrite\example@out
\newenvironment{example}%
 {\begingroup% Lets Keep the Changes Local
  \@bsphack
  \immediate\openout \example@out \jobname_ex.exa
  \let\do\@makeother\dospecials\catcode`\^^M\active
  \def\verbatim@processline{%
    \immediate\write\example@out{\the\verbatim@line}}%
  \verbatim@start}%
 {\immediate\closeout\example@out\@esphack\endgroup%
 % \par\small\addvspace{3ex plus 1ex}\vskip -\parskip
 \setlength{\parindent}{0pt}
    \setlength{\parskip}{1ex plus 0.5ex minus 0.7ex}
  \noindent
  \makebox[0.45\linewidth][t]{%
  \hspace*{+2ex}
  \setlength{\parindent}{0pt}
    \setlength{\parskip}{1ex plus 0.3ex minus 0.7ex}
  \begin{minipage}[t]{0.45\linewidth}
   % \vspace*{-1ex}
   \vspace*{-3.5ex}
   \refstepcounter{example}
   \captionsetup{options=out}
   \lstset{%
   caption=Entrada,basicstyle=\tiny\ttfamily\bf,language={[LaTeX]TeX}, numbersep=5mm, numbers=left, numberstyle=\tiny, % number style
   breaklines=true,framexleftmargin=10mm, xleftmargin=10mm,
   backgroundcolor=\color{mybrown!5},frameround=fttt,escapeinside=??,
   rulecolor=\color{mybrown!40!black},
   morekeywords={% Give key words here                                         % keywords
       maketitle},
   keywordstyle=\color[rgb]{0,0,1},                    % keywords
           commentstyle=\color[rgb]{0.133,0.545,0.133},    % comments
           stringstyle=\color[rgb]{0.627,0.126,0.941}  % strings
   %columns=fullflexible   
   }
    \lstinputlisting[]{\jobname_ex.exa}
         \end{minipage}}
  \hfill
  \hspace{30pt}%
   \setlength{\parindent}{0pt}
    \setlength{\parskip}{1ex plus 0.5ex minus 0.7ex}
  \makebox[0.5\linewidth][t]{%
  %\colorbox{ptcbackground!60}{
  \setlength{\parindent}{0pt}
    \setlength{\parskip}{1ex plus 0.3ex minus 0.7ex}
      \begin{minipage}[t]{0.4\linewidth}
   % \vspace*{0.5ex}
    \setlength{\parindent}{0pt}
      \noindent
  \setlength{\parskip}{1ex plus 0.4ex minus 0.7ex}
  \vspace*{-3.5ex} 
        \begin{trivlist}
             \scriptsize\bf\ttfamily \item \begin{out}
             {Pdf}\input{\jobname_ex.exa}\end{out}
    \end{trivlist}
      \end{minipage}%}
      }
      \par\addvspace{3ex plus 1ex}\vskip -\parskip
}
%%%%%%%%%%%%%%%%%%%%%%%%
\DeclareCaptionFormat{salid}{\colorbox{thered!40!black}{\parbox{\dimexpr\textwidth-2\fboxsep\relax}{\color{white}Ejercicio \thesection .\ 
\thesalida\hfill #3}}}
\captionsetup[salid]{format=salid,labelfont=white,textfont=white, singlelinecheck=false, margin=0pt, font={bf,footnotesize}}
\xcolortheorem[background=mybrown!5 ,titlebackground=mybrown!40!black ,titleboxcolor=mybrown!40!black ,boxcolor=mybrown!40!black]{salid}{Salida}
\newcounter{salida}[section]
\newwrite\salida@out
\newenvironment{salida}%
  {\begingroup% Lets Keep the Changes Local
  \@bsphack
  \immediate\openout \salida@out \jobname_eps.tex
  \let\do\@makeother\dospecials\catcode`\^^M\active
  \def\verbatim@processline{%
    \immediate\write\salida@out{\the\verbatim@line}}%
  \verbatim@start}%
 {\immediate\closeout\salida@out\@esphack\endgroup%
 \immediate\write18{pdflatex \jobname_eps.tex \jobname_eps.pdf }
 \immediate\write18{pdf2ps \jobname_eps.pdf \jobname_eps.eps }
    
  \setlength{\parindent}{0pt}
    \setlength{\parskip}{1ex plus 0.3ex minus 0.7ex}
     \begin{minipage}[t]{0.45\linewidth}
   % \vspace*{-1ex}
   %\vspace*{-3.5ex}
   \refstepcounter{salida}
   \captionsetup{options=salid}
   \lstset{%
   caption=Entrada,basicstyle=\tiny\ttfamily\bf,language={[LaTeX]TeX}, numbersep=5mm, numbers=left, numberstyle=\tiny, % number style
   breaklines=true,framexleftmargin=10mm, xleftmargin=10mm,
   backgroundcolor=\color{Tan!5},frameround=fttt,escapeinside=??,
   rulecolor=\color{thered!40!black},
   morekeywords={% Give key words here                                         % keywords
       maketitle},
   keywordstyle=\color[rgb]{0,0,1},                    % keywords
           commentstyle=\color[rgb]{0.133,0.545,0.133},    % comments
           stringstyle=\color[rgb]{0.627,0.126,0.941}  % strings
   %columns=fullflexible   
   }
    \lstinputlisting[]{\jobname_eps.tex}
         \end{minipage}
  %\hspace{30pt}%
   % \setlength{\parindent}{0pt}
    %\setlength{\parskip}{1ex plus 0.4ex minus 0.2ex}
    \hspace{20pt}
     \begin{minipage}[t]{0.40\linewidth}
   \begin{figure}[H]
\fcolorbox{thered!40!black}{thered!5}{\includegraphics[scale=0.4]{\jobname_eps.eps}}\end{figure}\end{minipage}
}
% }

  
%}
\DeclareCaptionFormat{prob}{\colorbox{thered!40!black}{\parbox{\dimexpr\textwidth-2\fboxsep\relax}{\color{white}Ejercicio \thesection .\ 
\theprob\hfill #3}}}
\captionsetup[prob]{format=prob,labelfont=white,textfont=white, singlelinecheck=false, margin=0pt, font={bf,footnotesize}}
\xcolortheorem[background=mybrown!5 ,titlebackground=mybrown!40!black ,titleboxcolor=mybrown!40!black ,boxcolor=mybrown!40!black]{prob}{Salida}
\newcounter{problema}[section]
\newwrite\problema@out
\newenvironment{problema}%
 {\begingroup% Lets Keep the Changes Local
  \@bsphack
  \immediate\openout \problema@out \jobname_pdf.tex
  \let\do\@makeother\dospecials\catcode`\^^M\active
  \def\verbatim@processline{%
    \immediate\write\problema@out{\the\verbatim@line}}%
  \verbatim@start}%
 {\immediate\closeout\problema@out\@esphack\endgroup%
 \immediate\write18{pdflatex \jobname_pdf.tex \jobname_pdf.pdf }
   \par\small\addvspace{3ex plus 1ex}\vskip -\parskip
\noindent
    \vspace*{-2ex}%
  \makebox[0.4\linewidth][l]{%
  \begin{minipage}[t]{0.9\linewidth}
  \captionsetup{options=prob}
   \lstset{%
   caption=Entrada,basicstyle=\tiny\ttfamily\bf,language={[LaTeX]TeX}, numbersep=5mm, numbers=left, numberstyle=\tiny, % number style
   breaklines=true,framexleftmargin=10mm, xleftmargin=10mm,
   backgroundcolor=\color{mybrown!5},frameround=fttt,escapeinside=??,
   rulecolor=\color{mybrown!40!black},
   morekeywords={% Give key words here                                         % keywords
       maketitle},
   keywordstyle=\color[rgb]{0,0,1},                    % keywords
           commentstyle=\color[rgb]{0.133,0.545,0.133},    % comments
           stringstyle=\color[rgb]{0.627,0.126,0.941}  % strings
   %columns=fullflexible   
   }
  
        \lstinputlisting[]{\jobname_pdf.tex}
        \end{minipage}
       }
 \vspace{20pt}

 \IfFileExists{\jobname_pdf.pdf}{% Si el fichero existe
 \begin{minipage}[t]{0.8\linewidth}
      \setlength{\parindent}{0pt}
    \setlength{\parskip}{1ex plus 0.4ex minus 0.2ex}
    \begin{figure}[H]
    \centering
    \caption{Salida}
  \fcolorbox{mybrown!40!black}{mybrown!5}{ \includegraphics[scale=0.4]{\jobname_pdf.pdf}}
\end{figure}\end{minipage}
}


  \par\addvspace{3ex plus 1ex}\vskip -\parskip
}

\newcommand{\figcaption}[1]{\def\@captype{figure}\caption{#1}}
 \renewcommand\@seccntformat[1]%
{\color{green}\csname the#1\endcsname.\quad}
\newcommand{\helv}{\fontfamily{phv}\fontsize{9}{11}\selectfont}
\newcommand{\helvi}{\fontfamily{phv}\fontseries{b}\fontsize{9}{11}\selectfont}
%\newcounter{notas}
% \newcommand{\notas }{\stepcounter{notas}\vskip 6pt \colorbox{red}{\thechapter.\thenotas.\color{blue}{Nota:}\vskip 6pt}
%  }
\newcommand{\margen }[1]{\marginpar{\parbox{4cm}{\small\emph{#1}}}}
\newenvironment{marnota}[1]{\begin{minipage}{4cm}\small\emph{#1}
}
{\end{minipage}
}
\newcommand{\pie}[1]{\begin{changemargin}{6cm}{5cm}{-0.1cm}{5cm}{2cm}#1\end{changemargin}}
%\renewenvironment{code}{\begin{quote}}{\end{quote}}
\newcommand{\cih}[1]{%
\index{instrucciones!#1@\texttt{\bs#1}}%
\index{#1@\texttt{\hspace*{-1.2ex}\bs #1}}}
\newcommand{\ci}[1]{\cih{#1}\texttt{\bs#1}}
%Package
\newcommand{\pai}[1]{%
\index{paquetes!#1@\textsf{#1}}%
\index{#1@\textsf{#1}}%
\textsf{#1}}
% Entrada en el 'indice de entorno
\newcommand{\ei}[1]{%
\index{entornos!\texttt{#1}}%
\index{#1@\texttt{#1}}%
\texttt{#1}}
% Entrada en el 'indice para mensajes
\newcommand{\wni}[1]{%
\index{mensaje!\texttt{#1}}%
\texttt{#1}}
% Entrada en el 'indice de una palabra
\newcommand{\wi}[1]{\index{#1}#1}
%
% Instrucciones de composici'on
%
\newenvironment{command}%
    {\nopagebreak\par\small\addvspace{3.2ex plus 0.8ex minus 0.2ex}%
     \vskip -\parskip
     \noindent%
      \setlength{\arrayrulewidth}{1mm}
          \arrayrulecolor{mybrown!40!black}
     \begin{tabular}{|l|}\rowcolor{mybrown!5}\hline\rule{0pt}{1em}\ignorespaces}%
    {\\\hline\end{tabular}\par\nopagebreak\addvspace{3.2ex plus 0.8ex
        minus 0.2ex}%
     \vskip -\parskip}
%
% Composici'on de fragmentos de c'odigo
%
\newenvironment{code}%
    {\nopagebreak\par\small\addvspace{3.2ex plus 0.8ex minus 0.2ex}%
     \vskip -\parskip
     \noindent%
      \setlength{\arrayrulewidth}{1mm}
          \arrayrulecolor{thered!40!black}
     \begin{tabular}{|l|}\rowcolor{Tan!5}\hline\rule{0pt}{1pt}\ignorespaces}%
    {\\\hline\end{tabular}\par\nopagebreak\addvspace{3.2ex plus 0.8ex
        minus 0.2ex}%
     \vskip -\parskip}
% \cvhrulefill{<color>}{<thickness>}
\newcommand*\cvhrulefill[2]{%
  \leavevmode\color{#1}\leaders\hrule\@height#2\hfill \kern\z@\normalcolor}
% \crule{<color>}{<width>}{<thickness>}
\newcommand\crule[3]{%
  \color{#1}\rule{#2}{#3}\normalcolor}
\newcommand{\codigo}[1]{\lstinline!\\#1!}
% Entorno Intro
\newenvironment{intro}{\sffamily}{\vspace*{2ex minus 1.5ex}}
%%%%%lined
\NewEnviron{lined}[1]%
 {\begin{center}
  \begin{minipage}{#1}\crule{red!40!black}{#1 +0.1\linewidth}{2pt}\vspace{2ex}
   \BODY 
   \crule{red!40!black}{#1 +0.1\linewidth}{2pt}
   \end{minipage}\vspace{2ex}
 \end{center}\vspace{2ex}}
%%%%%%
\protected\def\PdfLaTeX{P\kern -.15em\raisebox{-0.21em}{D}\kern -.05em F\LaTeX}
\protected\def\PdfTeX{P\kern -.15em\raisebox{-0.21em}{D}\kern -.05em F\TeX}
\newenvironment{lcpar}{%
\begingroup
\setlength{\leftskip}{0pt plus 1fil}%
\setlength{\rightskip}{-\leftskip}%
\setlength{\parfillskip}{0pt plus 2fil}
}{%
\par\endgroup
}
\renewcommand*{\LettrineFontHook}
{\bfseries}
\renewcommand*{\LettrineTextFont}
{\bfseries}
%%%%%%%%%%%%%%%%%%%%%%%%%%%%%%%%%%%%%%%%%
%%%%%%%%%%%%%%%%%%%%%%%%%%%%%%%%%
%%%%%%%%%%%%%%%%%%%%%%%%%%%%%%%%%%%%%%%%%%
\tikzstyle{mybox} = [draw=red!40!black, fill=mybrown!10, very thick,
    rectangle, rounded corners, inner sep=10pt, inner ysep=20pt]
\tikzstyle{fancytitle} =[fill=red!40!black, text=white]
\NewEnviron{competencias}{\vskip 5pt
\begin{tikzpicture}

% First box
\node [mybox] (box1){%
    \begin{minipage}{\textwidth}
      \BODY
    \end{minipage}
    };
    \node[fancytitle, rounded corners] at (box1.north) {\Large Objetivos};
\end{tikzpicture}
 }
%%%%%%%%%%%%%%%%%%%%%%
\NewEnviron{logros}{\vskip 5pt
\begin{tikzpicture}

% First box
\node [mybox] (box1){%
    \begin{minipage}{\textwidth}
      \BODY
    \end{minipage}
    };
    \node[fancytitle, rounded corners] at (box1.north) {\Large Indicadores de Logros
};
\end{tikzpicture}
 }
%%%%%%%%%%%%%%%%%%%%%%%%%%%%%%%%%%%%%%%%
%%%%%%%%%%%%%%%%%%%%%%%%%%%%%%%%
%\newcounter{ideas}
%%\newboxedtheorem[title=Definici\'on.\thesection . \thedefinicionn, labelbox=, boxcolor=caper!75!black  ,background=caper!5  ,titleboxcolor=caper!75!black  , titlebackground=caper!60 ]{definicionn}{Definici\'on}
%%\newboxedtheorem[title=T\'ermino.\thesection . \theideas, labelbox=, boxcolor=rhodo!75!black ,background=rhodo!5  ,titleboxcolor=rhodo!75!black, titlebackground=rhodo!60]{ideas}{T\'ermino}
%%%%%%%%%%%%%%%%%%%%%%%%%%%%%%%%
%%\newboxedtheorem[title=T\'ermino no defnido.\thesection . \thetndefinido, labelbox=, boxcolor=MilkTea ,background=ptcbackground!60  ,titleboxcolor=MilkTea , titlebackground=ptctitle]{tndefinido}{T\'ermno no definido}
%%%%%%%%%%%%%%%%%%%%%%%%%%%%%%
\newboxedtheorem[title=Definici\'on.\ \thesection . \thedefinicionn , labelbox= ,boxcolor=caper!75!black,background =caper!5,titleboxcolor=caper!75!black,titlebackground=caper!60]{definicionn}%{Teorema}
  %%%%%%%%%%%%%%%
 \newboxedtheorem[title=T\'ermino.\ \thesection . \theideas, labelbox= ,boxcolor=doc!75!black,background =doc!5 ,titleboxcolor=caper!75!black,titlebackground=doc!60]{ideas}%{Teorema}
 %%%%%%%%%%%%%%
  % % % % % % % %
  \newboxedtheorem[title=T\'ermino no defnido. \thesection . \thetndefinido, labelbox= ,boxcolor=MilkTea,background = ptcbackground!60,titleboxcolor=MilkTea,titlebackground=ptctitle]{tndefinido}%{T\'erminos no definidos}
  % % % % % % % %
  \newboxedtheorem[title=Ejemplo. \thesection . \theejem, labelbox= ,boxcolor=ptctitle!75!black,background = ptcbackground!60,titleboxcolor=ptctitle!60!black,titlebackground=ptctitle]{ejem}%{T\'erminos no definidos}
  %%%%%%%%%%%%%%%%%%%
 
 \newboxedtheorem[title=Ejemplos. \thesection . \theejems, labelbox= ,boxcolor=ptctitle!75!black,background = ptcbackground!60,titleboxcolor=ptctitle!60!black,titlebackground=ptctitle]{ejems}%{T\'erminos no definidos}
 
%%%%%%%%%%%%%%%%%%%%%%%  
 
% \newcounter{axioma}
%\newenvironment{axioma}[2]{\vskip 5pt
%\refstepcounter{axioma}
%    \begin{tcolorbox}[colback=mybrown!5,colframe=mybrown!40!black,title=Axioma.\thechapter.\thepostulado  \ \bf{#2}]
% #1
%\end{tcolorbox}\index{Axioma!#2}            }{
%
%                \vskip 5pt
% }
 \newboxedtheorem[title=Axioma.\ \thesection . \theaxioma, labelbox= ,boxcolor=mybrown!40!black,background =mybrown!5 ,titleboxcolor=mybrown!75!black,titlebackground=mybrown!60]{axioma}%{Teorema}
 \newboxedtheorem[title=Leyes de inferencia.\ \thesection . \theley, labelbox= ,boxcolor=doc!75!black,background =doc!5 ,titleboxcolor=caper!75!black,titlebackground=doc!60]{ley}%{Teorema}
 \newboxedtheorem[title=Lema.\ \thesection . \thelema, labelbox= ,boxcolor=doc!75!black,background =doc!5 ,titleboxcolor=caper!75!black,titlebackground=doc!60]{lema}%{Teorema}
 \newboxedtheorem[title=Teorema.\ \thesection . \theteorema, labelbox= ,boxcolor=mybrown!40!black,background =mybrown!5 ,titleboxcolor=mybrown!75!black,titlebackground=mybrown!60]{teorema}%{Teorema}
  \newboxedtheorem[title=Operaci\'on\ \thesection . \theoperacion, labelbox= ,boxcolor=doc!75!black,background =doc!5 ,titleboxcolor=caper!75!black,titlebackground=doc!60]{operacion}%{Teorema}
 \newcommand\HUGE{\@setfontsize\Huge{38}{47}}
\newcommand\HHUGE{\@setfontsize\HHUGE{58}{67}}
\newcommand\peque{\@setfontsize\peque{8}{9}}

 \makeatother