\renewcommand\thesection{PC\ \nplpadding{2}\numprint{\arabic{section}}}
\pagestyle{eimat}
\pagecolor{ptcbackground}
\chapter{Posters}
\thispagestyle{empty}
\pagecolor{ptcbackground}
\chaptertoc
%%%%%%%%%%%%%%%%%%%%1 %%%%%%%%%%%%%%%%%
\author{%
\tema{Teor\i'a de Operadores,}\vspace{2pt}\\
   Germ\'an Fabian Escobar Fiesco,\vspace{2pt} \\
    Universidad Surcolombiana,\vspace{2pt} \\
    \hspace*{-2cm}\texttt{\scriptsize dacoros@gmail.com }\vspace{20pt} \\
   Jessica Vizcaya Garz\'on,\vspace{2pt}\\
    Ang\'elica Narv\'aez,\vspace{2pt}\\
   Karen Yulier Montealegre,\vspace{2pt}\\
    Daniela Cortes Ospina ,\vspace{2pt}\\
      Vicente Alvarez Arias,\vspace{2pt}\\
    Estudiantes de la  Universidad Surcolombiana\vspace{2pt} \\ 
    %\hspace*{-2cm}\texttt{\scriptsize lhhurtado@uniquindio.edu.co}\\
         }
\pagecolor{white}
\pagestyle{eimat}
%\setcounter{section}{2}
\begin{titlepage}
\pagecolor{white}
%%%%%%%definiciones
\newcommand{\R}{\ensuremath{\mathbb{R}}}
%%%%%%%%%%
%\pagecolor{yellow}\'afterpage{\nopagecolor}
%\pagestyle{plain}
\BgThispage
\newgeometry{left=2cm,right=2cm,top=2cm,bottom=2cm}
\vspace*{-1.1cm}
\noindent
\def\titulo#1{\section{#1}}

\section{\bf\large\textcolor{white}{Marcos en Espacios de Hilbert y su extensui\'on a  Espacios de Krein}}
\vspace*{2cm}\par
\noindent

\begin{minipage}{0.5\linewidth}
\begin{minipage}{0.45\linewidth}
    \begin{flushright}
        \printauthor
    \end{flushright}
\end{minipage} \hspace{-3pt}
%
\begin{minipage}{0.02\linewidth}
   \color{ptctitle} \rule{1pt}{225pt}
\end{minipage} 
\end{minipage}
\hspace*{-4.5cm}
%
\begin{minipage}{0.85\linewidth}
\begin{minipage}{0.85\linewidth}
\footnotesize
\vspace{5pt}
    \begin{resumen}
   Esta propuesta de investigaci\'on tiene como prop\'osito estudiar y profundizar las propiedades m\'as relevantes de los marcos en los espacios de Hilbert y su extensi\'on a espacios Krein; ya que; conocer la extensi\'on de marco en un espacio de Krein, as\'i como sus propiedades es importante porque se requiere profundizar este tema debido al acelerado desarrollo elaborado por grandes investigadores en diferentes areas de la matem\'atica. Hoy en d\'ia existe una extensa lista de aplicaciones de la teor\'ia de marcos como lo son: la computaci\'on cu\'antica, el an\'alisis de multiresoluci\'on,codificaci\'on de antena m\'ultiple, teor\'ia de muestreo, procesamiento de se\~nales e im\'agenes, comprensi\'on de datos, entre otros, por tal motivo consideramos esta tem\'atica de gran inter\'es en el \'area de la matem\'atica, as\'i como sus aplicaciones a las diferentes ramas de las ciencias naturales y exactas.
    \end{resumen}
    %\hspace*{25pt}\palabras{one, two, three, four}
\end{minipage}
\vspace*{5pt}\\
\footnotesize
%\begin{minipage}{0.85\linewidth}
%\vspace{5pt}
%    \begin{abstract} 
%  
%\lipsum[3-4]
%    \end{abstract}\vspace{5pt}
%     \hspace*{25pt}\keywords{one, two, three, four}
    
%\end{minipage}
\end{minipage}
\vspace{5pt}
\begin{thebibliography}{99}

\bibitem{Ad}{\sc KREISZIG E.} (1989) {\it INTRODUCTORY FUNCTIONAL ANALYSIS WITH APPLICATIONS}. Wiley Classics Library Edition Published 1989, New York, EEUU.
\bibitem{bav1} {\sc BOGNAR, J.} (1974) Indefinite inner product spaces. \emph{Springer, Berlin. }
\end{thebibliography}
\end{titlepage}
%%%%%%%%%%%%%%%%%%%%%%%% 5
\setcounter{section}{4}
\author{%
\tema{An\'alisis,}\vspace{2pt}\\
  Alex F. Aristizabal,\vspace{2pt} \\
    %Universidad Surcolombiana,\vspace{2pt} \\
    \hspace*{-2cm}\texttt{\scriptsize afad1991@gmail.com}\vspace{10pt} \\
   Pedro L. Hern\'andez,\vspace{2pt}\vspace{2pt}\\
   \hspace*{-2cm}\texttt{\scriptsize phernandezllanos@gmail.com}\vspace{10pt}\\
    Luis R. Siado,\vspace{2pt}\\
    \hspace*{-2cm}\texttt{\scriptsize luisrsiado88@gmail.com}\vspace{10pt}\\
  Alejandro Urieles ,\vspace{2pt}\\
   \hspace*{-2cm}\texttt{\scriptsize aurielesg@gmail.com}\vspace{10pt}\\
       Programa de Matem\'aticas,\vspace{2pt}\\ Universidad del Atl\'antico,\vspace{2pt}\\
        Barranquilla, Colombia\\ \vspace{2pt} 
    %\hspace*{-2cm}\texttt{\scriptsize lhhurtado@uniquindio.edu.co}\\
         }
\pagecolor{white}
\pagestyle{eimat}
%\setcounter{section}{2}
\begin{titlepage}
\pagecolor{white}
%%%%%%%definiciones
\newcommand{\R}{\ensuremath{\mathbb{R}}}
%%%%%%%%%%
%\pagecolor{yellow}\'afterpage{\nopagecolor}
%\pagestyle{plain}
\BgThispage
\newgeometry{left=2cm,right=2cm,top=2cm,bottom=2cm}
\vspace*{-1.1cm}
\noindent
\def\titulo#1{\section{#1}}

\section{\bf\large\textcolor{white}{Estudio sobre los polinomios ortogonales de Jacobi y Gegenbauer. Algunas propiedades}}
\vspace*{2cm}\par
\noindent

\begin{minipage}{0.5\linewidth}
\begin{minipage}{0.45\linewidth}
    \begin{flushright}
        \printauthor
    \end{flushright}
\end{minipage} \hspace{-3pt}
%
\begin{minipage}{0.02\linewidth}
   \color{ptctitle} \rule{1pt}{225pt}
\end{minipage} 
\end{minipage}
\hspace*{-4.5cm}
%
\begin{minipage}{0.85\linewidth}
\begin{minipage}{0.85\linewidth}
\footnotesize
\vspace{5pt}
    \begin{resumen}
  Nuestro trabajo se basa en el estudio de los polinomios ortogonales de Jacobi  $P_{n}^{(\alpha , \beta)}(x)$ con $\alpha,\beta>-1$, $x\in[-1,1]$ y Gegenbauer $P_{n}^{(\lambda)}(x)$ con $\lambda>-\frac{1}{2}$ y $x\in[-1,1]$. Abordaremos propiedades algebraicas y diferenciales como la ecuaci\'on diferencial que satisfacen, f\'ormula de Rodrigues, ortogonalidad, norma, coeficiente principal, f\'ormula de recurrencia a tres t\'erminos, su representaci\'on explicita a trav\'es de la funci\'on hipergeom\'etrica y la f\'ormula de Christofel-Darboux.
    \end{resumen}
    %\hspace*{25pt}\palabras{one, two, three, four}
\end{minipage}
\vspace*{5pt}\\
\footnotesize
%\begin{minipage}{0.85\linewidth}
%\vspace{5pt}
%    \begin{abstract} 
%  
%\lipsum[3-4]
%    \end{abstract}\vspace{5pt}
%     \hspace*{25pt}\keywords{one, two, three, four}
    
%\end{minipage}
\end{minipage}
\vspace{5pt}
\begin{thebibliography}{99}
\bibitem{Ad}{\sc Chiara, T. S.} (1978) {\it An Introducci\'on to Orthogonal Polynomials}.  Gordon and Breach,Science Publisher Inc., New York, EEUU.

%\bibitem{bav1} {\sc Lopez Lagomasino, G. and \sc Pijeira, H.} (2001) {\it Polinomios Ortogonales}.  XIV Escuela Venezolana de Matem\'aticas, M\'erida, Venezuela.%

\bibitem{bav1} {\sc Szeg\"o, G.} (2001) {\it Orthogonal Polynomials}. American Mathematical Society, Providence, Rhode Island, EEUU.

\bibitem{bav1} {\sc Marcell\'an, F. ,\sc Quintana, Y. and \sc Urieles, A.} (2013) {\it On the Pollard decomposition method applied to some Jacobi - Sobolev expansions}. Turk J Math
(2013) 37: 934 - 948, doi:10.3906/mat-1208-29.

\end{thebibliography}
\end{titlepage}
%%%%%%%%%%%%%%%%%%%%10 %%%%%%%%%%%%%%%%%
\setcounter{section}{9}
\author{%
\tema{Matematica Aplicada,}\vspace{2pt}\\
   Camilo Barrios C,\vspace{2pt} \\
    %Universidad Surcolombiana,\vspace{2pt} \\
    \hspace*{-2cm}\texttt{\scriptsize cbarrioscamargo94@gmail.com }\vspace{20pt} \\
   Valbuena D. S,\vspace{2pt}\\
   \hspace*{-2cm}\texttt{\scriptsize svalbuen1@gmail.com }\vspace{10pt} \\
      Universidad  del Atl\'antico,\vspace{2pt} \\ 
       Barranquilla. Colombia \vspace{2pt} \\ 
    %\hspace*{-2cm}\texttt{\scriptsize lhhurtado@uniquindio.edu.co}\\
         }
\pagecolor{white}
\pagestyle{eimat}
%\setcounter{section}{2}
\begin{titlepage}
\pagecolor{white}
%%%%%%%definiciones
\newcommand{\R}{\ensuremath{\mathbb{R}}}
%%%%%%%%%%
%\pagecolor{yellow}\'afterpage{\nopagecolor}
%\pagestyle{plain}
\BgThispage
\newgeometry{left=2cm,right=2cm,top=2cm,bottom=2cm}
\vspace*{-1.1cm}
\noindent
\def\titulo#1{\section{#1}}

\section{\bf\large\textcolor{white}{Uso de la programaci\'on cient\'ifica para estimar el \'area bajo la curva de una funci\'on en un dominio dado }}
\vspace*{2cm}\par
\noindent

\begin{minipage}{0.5\linewidth}
\begin{minipage}{0.45\linewidth}
    \begin{flushright}
        \printauthor
    \end{flushright}
\end{minipage} \hspace{-3pt}
%
\begin{minipage}{0.02\linewidth}
   \color{ptctitle} \rule{1pt}{225pt}
\end{minipage} 
\end{minipage}
\hspace*{-4.5cm}
%
\begin{minipage}{0.85\linewidth}
\begin{minipage}{0.85\linewidth}
\footnotesize
\vspace{5pt}
    \begin{resumen}
  Con este proyecto se pretende mostrar el uso de la programaci\'on cient\'ifica a la matem\'atica, siendo un proyecto de aula realizado en el curso de programaci\'on de computadores II del Programa de Matem\'aticas. El objetivo de este trabajo fue dise\~nar e implementar  un programa computacional haciendo uso de Matlab\circledR, para  aproximar el \'area bajo la curva $f(x)$ en el intervalo $\left[ a, b\right] $ , con $f(x)$ continua en dicho intervalo, para esta estimaci\'on se utiliz\'o el m\'etodo de los trapecios, mostrando as\'i la importancia de la programaci\'on en los m\'etodos num\'ericos y en las matem\'aticas, se implementaron  adem\'as  procesos para calcular el error absoluto y relativo porcentual de la aproximaci\'on obtenida con este m\'etodo, y otro que grafique el \'area bajo la curva y la gr\'afica del error para mayor comprensi\'on del m\'etodo y del estimativo obtenido.
La implementaci\'on computacional fue realizada usando programaci\'on estructurada adicionalmente se incorpor\'o el manejo de la interface gr\'afica para los usuarios del programa computacional (GUIs).

    \end{resumen}
    \hspace*{25pt}\palabras{\'area bajo una curva, regla del trapecio, Matlab\circledR, programaci\'on cient\'ifica, GUIs.
}
\end{minipage}
\vspace*{5pt}\\
\footnotesize
%\begin{minipage}{0.85\linewidth}
%\vspace{5pt}
%    \begin{abstract} 
%  
%\lipsum[3-4]
%    \end{abstract}\vspace{5pt}
%     \hspace*{25pt}\keywords{one, two, three, four}
    
%\end{minipage}
\end{minipage}
\vspace{5pt}
\begin{thebibliography}{99}

\bibitem{Edi}{\sc UDIMAMBER, J., O.Santos, R. Fabregat.}(2009) {\it  Introduction to Algorithms: a creative approach. Addison-Wesley Publishing Company}. 

\bibitem{Demmel} {\sc DEMMEL JAMES W.} (2010){\it Applied Numerical Linear Algebra, MIT, SIAM.}.
\end{thebibliography}
\end{titlepage}

%%%%%%%%%%%%%%%%%%% 11 %%%%%%%%%%%%
\author{%
\tema{An\'alisis,}\vspace{2pt}\\
   Pedro L. Hern\'andez,\vspace{2pt} \\
    Programa de Matem\'aticas,\vspace{2pt} \\
     Universidad del Atl\'antico,\vspace{2pt} \\ 
     Barranquilla, Colombia,\vspace{2pt} \\
    \hspace*{-2cm}\texttt{\scriptsize phernandezllanos@gmail.com }\vspace{10pt} \\
   William D. Ramirez,\vspace{2pt}\\
   Departamento de Matem\'aticas Pura y Aplicada,\vspace{2pt}\\ 
   Universidad Sim\'on Bolivar,\vspace{2pt}\\
    Caracas 1080 A, Venezuela,\vspace{2pt}\\
      \hspace*{-2cm}\texttt{\scriptsize lic.williamramirezquiroga@gmail.com }\vspace{10pt} \\
   Alejandro Urieles,\vspace{2pt}\\
   Programa de Matem\'aticas,\vspace{2pt} \\
     Universidad del Atl\'antico,\vspace{2pt} \\ 
     Barranquilla, Colombia,\vspace{2pt} \\
    \hspace*{-2cm}\texttt{\scriptsize aurielesg@gmail.com }\vspace{20pt} \\
      %\hspace*{-2cm}\texttt{\scriptsize lhhurtado@uniquindio.edu.co}\\
         }
\pagecolor{white}
\pagestyle{eimat}
%\setcounter{section}{2}
\begin{titlepage}
\pagecolor{white}
%%%%%%%definiciones
\newcommand{\R}{\ensuremath{\mathbb{R}}}
%%%%%%%%%%
%\pagecolor{yellow}\'afterpage{\nopagecolor}
%\pagestyle{plain}
\BgThispage
\newgeometry{left=2cm,right=2cm,top=2cm,bottom=2cm}
\vspace*{-1.1cm}
\noindent
\def\titulo#1{\section{#1}}

\section{\bf\large\textcolor{white}{Sobre la Funci\'on Zeta Lerch y los polinomios de Apostol-Bernoulli }}
\vspace*{2cm}\par
\noindent

\begin{minipage}{0.5\linewidth}
\begin{minipage}{0.45\linewidth}
    \begin{flushright}
        \printauthor
    \end{flushright}
\end{minipage} \hspace{-3pt}
%
\begin{minipage}{0.02\linewidth}
   \color{ptctitle} \rule{1pt}{325pt}
\end{minipage} 
\end{minipage}
\hspace*{-4.5cm}
%
\begin{minipage}{0.85\linewidth}
\begin{minipage}{0.85\linewidth}
\footnotesize
\vspace{5pt}
    \begin{resumen}
  Nuestro trabajo se basa en un estudio de la funci\'on Zeta Lerch $\displaystyle \phi(x,a,s)$, donde $Re(s)>1 $, x real y $a$ un entero positivo y los polinomios de Apostol-Bernoulli $\mathfrak{B}_{n}(x;\lambda)$ con $\lambda \in \mathbb{C}$, donde analizaremos la funci\'on Zeta de Riemann $\zeta (s)$, siendo  $s=\sigma + ti$, la funci\'on Zeta de Hurwitz $\zeta (s,a)$ para $s=\sigma + ti$ y $0<a\leq 1$, la funci\'on Zeta Lerch $\displaystyle \phi(x,a,s)$ y sus respectivas propiedades. Finalmente se mostrar\'an algunos resultados sobre la relaci\'on que hay entre la funci\'on Zeta Lerch $\displaystyle \phi(x,a,s)$ y los polinomios de Bernoulli $\mathfrak{B}_{n}(x;\lambda)$. 
    \end{resumen}
    %\hspace*{25pt}\palabras{one, two, three, four}
\end{minipage}
\vspace*{5pt}\\
\footnotesize
%\begin{minipage}{0.85\linewidth}
%\vspace{5pt}
%    \begin{abstract} 
%  
%\lipsum[3-4]
%    \end{abstract}\vspace{5pt}
%     \hspace*{25pt}\keywords{one, two, three, four}
    
%\end{minipage}
\end{minipage}
\vspace{5pt}
\begin{thebibliography}{99}

\bibitem{hzf} Apostol, T. \textit{"On the Lerch Zeta  function''}. Pacific J. Math.\textbf{1}, 161-167 (1951).
\bibitem{taps} 
%Apostol, T. %
\textit{Introduction to Analitic Number Theory}, Springer-Verlag, New York.(1976).
\bibitem{sri} Srivastava, H.M., Todorov, P.G.: \textit{"An Explicit Formula for the Generalized Bernoulli Polynomials''}, J. Mat. Anl. Appl. 130, 509-513 (1988).
\end{thebibliography}
\end{titlepage}
%%%%%%%%%%%%%%%%%%%%% 12
\author{%
\tema{Educaci{\'o}n matem{\'a}tica,}\vspace{2pt}\\
  Jazm{\'i}n Johanna ,\vspace{2pt} \\
        \hspace*{-2cm}\texttt{\scriptsize gjazmin\_725@hotmail.com }\vspace{10pt} \\
   Laura Tatiana Jaimes Izaquita,\vspace{2pt}\\
        \hspace*{-2cm}\texttt{\scriptsize laujaimes6@gmail.com }\vspace{10pt} \\
        Estudiantes de Licenciatura en Matem{\'a}ticas, UIS.\vspace{10pt} \\
   Luz Estella Giraldo,\vspace{2pt}\\
   Profesor titular, UIS,\vspace{2pt} \\
        \hspace*{-2cm}\texttt{\scriptsize luzestelagiraldo@yahoo.es }\vspace{20pt} \\
      %\hspace*{-2cm}\texttt{\scriptsize lhhurtado@uniquindio.edu.co}\\
         }
\pagecolor{white}
\pagestyle{eimat}
%\setcounter{section}{2}
\begin{titlepage}
\pagecolor{white}
%%%%%%%definiciones
\newcommand{\R}{\ensuremath{\mathbb{R}}}
%%%%%%%%%%
%\pagecolor{yellow}\'afterpage{\nopagecolor}
%\pagestyle{plain}
\BgThispage
\newgeometry{left=2cm,right=2cm,top=2cm,bottom=2cm}
\vspace*{-1.1cm}
\noindent
\def\titulo#1{\section{#1}}

\section{\bf\large\textcolor{white}{Desarrollo del pensamiento geom{\'e}trico en estudiantes de tercer grado desde el enfoque de resoluci{\'o}n de problemas con implementaci{\'o}n de las TIC }}
\vspace*{2cm}\par
\noindent

\begin{minipage}{0.5\linewidth}
\begin{minipage}{0.45\linewidth}
    \begin{flushright}
        \printauthor
    \end{flushright}
\end{minipage} \hspace{-3pt}
%
\begin{minipage}{0.02\linewidth}
   \color{ptctitle} \rule{1pt}{225pt}
\end{minipage} 
\end{minipage}
\hspace*{-4.5cm}
%
\begin{minipage}{0.85\linewidth}
\begin{minipage}{0.85\linewidth}
\footnotesize
\vspace{5pt}
    \begin{resumen}
 Se presenta una experiencia de aula realizada en una instituci{\'o}n educativa del sector oficial de Bucaramanga, en la cual se dise\~n{\'o} una propuesta para la ense\~nanza de las figuras y cuerpos geom{\'e}tricos en tercero primaria, que incorpora  tres actividades desde el enfoque de la resoluci{\'o}n de problemas, utilizando las TIC como herramientas de mediaci\'on en los procesos de ense\~nanza y aprendizaje.
\vspace{0.2cm}\\
La hip{\'o}tesis que gu{\'i}a este dise\~no es que a partir de la implementaci{\'o}n de tres actividades pensadas desde la cotidianidad, el estudiante pueda desarrollar las cuatro etapas que propone Polya (1989)  para la soluci{\'o}n de problemas matem{\'a}ticos: comprender el problema, configurar un plan, ejecutar el plan y mirar hacia atr\'as. La secuencia de aprendizaje que enmarca la propuesta integradora parte del siguiente eje problematizador: ¿C{\'o}mo ense\~nar figuras y cuerpos geom{\'e}tricos a estudiantes de tercero primaria para que puedan establecer una relaci{\'o}n significativa entre la geometr{\'i}a y el entorno que los rodea?
\vspace{0.2cm}\\
Etapa I. Comprender el problema: Dise\~no de un fotograma en Movie Maker. Disponible en: \url{https://www.youtube.com/watch?v=yulqetdlnqc.}\\
Etapa II. Configurar un plan: Se propone un cuento publicado en Storybird  que desarrolla una situaci{\'o}n matem{\'a}tica.\\
Etapa III. Ejecutar el plan: Creaci{\'o}n de una WebQuest.\\
Etapa VI. Mirar hacia atr{\'a}s: Evaluaci{\'o}n de resultados, ventajas y aspectos a mejorar.
\vspace{0.5cm}\\
Se destaca de esta experiencia de aula, el hecho de que el uso de las tecnolog{\'i}as mediante la supervisi\'on del docente, motiv{\'o} significativamente el aprendizaje de los conceptos geom{\'e}tricos que se quer\'ian ense\~nar.
    \end{resumen}
    %\hspace*{25pt}\palabras{one, two, three, four}
\end{minipage}
\vspace*{5pt}\\
\footnotesize
%\begin{minipage}{0.85\linewidth}
%\vspace{5pt}
%    \begin{abstract} 
%  
%\lipsum[3-4]
%    \end{abstract}\vspace{5pt}
%     \hspace*{25pt}\keywords{one, two, three, four}
    
%\end{minipage}
\end{minipage}
\vspace{5pt}
\begin{thebibliography}{99}
\bibitem{Ad}{\sc Polya, George.} (1989) {\it C{\'o}mo plantear y Resolver Problemas}.  M{\'e}xico. Editorial  Trillas S.A.
\end{thebibliography}
\end{titlepage}
%%%%%%%%%%%%%%%%%%%%% 13
\author{%
\tema{Matematica Aplicada,}\vspace{2pt}\\
  Theran S. L,\vspace{2pt} \\
          Laura Tatiana Jaimes Izaquita,\vspace{2pt}\\
        \hspace*{-2cm}\texttt{\scriptsize laujaimes6@gmail.com }\vspace{10pt} \\
        Leyva C. W,\vspace{2pt}\\
        Garcia O. A,\vspace{2pt}\\
        Racedo N. F,\vspace{2pt}\\
  Grupo GEOEL,\vspace{2pt}\\
   Prog. F\'isica,\vspace{2pt}\\
   Universidad del Atl\'antico,\vspace{2pt}\\
   Colombia,\vspace{2pt} \\
        \hspace*{-2cm}\texttt{\scriptsize fran@mail.uniatlantico.edu.co }\vspace{20pt} \\
         Valbuena D. S,\vspace{2pt}\\
         Prog. Matem\'aticas,\vspace{2pt}\\
          Universidad del Atl\'antico. Colombia\vspace{2pt}\\
      %\hspace*{-2cm}\texttt{\scriptsize lhhurtado@uniquindio.edu.co}\\
         }
\pagecolor{white}
\pagestyle{eimat}
%\setcounter{section}{2}
\begin{titlepage}
\pagecolor{white}
%%%%%%%definiciones
\newcommand{\R}{\ensuremath{\mathbb{R}}}
%%%%%%%%%%
%\pagecolor{yellow}\'afterpage{\nopagecolor}
%\pagestyle{plain}
\BgThispage
\newgeometry{left=2cm,right=2cm,top=2cm,bottom=2cm}
\vspace*{-1.1cm}
\noindent
\def\titulo#1{\section{#1}}

\section{\bf\large\textcolor{white}{Simulaci\'on em Matlab\circledR\ 
de la propagaci\'on de un haz laser usando diferencias finitas }}
\vspace*{2cm}\par
\noindent

\begin{minipage}{0.5\linewidth}
\begin{minipage}{0.45\linewidth}
    \begin{flushright}
        \printauthor
    \end{flushright}
\end{minipage} \hspace{-3pt}
%
\begin{minipage}{0.02\linewidth}
   \color{ptctitle} \rule{1pt}{280pt}
\end{minipage} 
\end{minipage}
\hspace*{-4.5cm}
%
\begin{minipage}{0.85\linewidth}
\begin{minipage}{0.85\linewidth}
\footnotesize
\vspace{5pt}
    \begin{resumen}
 
Se describen los fundamentos matem\'aticos del m\'etodo de diferencias finitas, MDF, y su implementaci\'on en Matlab\circledR\ para simular el comportamiento de un haz de luz l\'aser  propag\'andose en diferentes medios. Se parte de las ecuaciones de Maxwell en su formulaci\'on original y se procede a su discretizaci\'on para continuar con la implementaci\'on del algoritmo computacional en Matlab.

En el c\'odigo implementado se pueden realizar modificaciones simples que permiten simular diferentes situaciones del campo de la f\'isica. En particular se analiza el comportamiento de un haz laser. Se destaca la convergencia de los resultados obtenidos con resultados similares reportados en la literatura.
    \end{resumen}
    \hspace*{25pt}\palabras{Diferencias finitas; electromagnetismo; simulaci\'on; Matlab\circledR.}
\end{minipage}
\vspace*{5pt}\\
\footnotesize
%\begin{minipage}{0.85\linewidth}
%\vspace{5pt}
%    \begin{abstract} 
%  
%\lipsum[3-4]
%    \end{abstract}\vspace{5pt}
%     \hspace*{25pt}\keywords{one, two, three, four}
    
%\end{minipage}
\end{minipage}
\vspace{5pt}
\begin{thebibliography}{99}
\bibitem{1}{\sc Weideman, J.A.C. and Reddy, S.C.}(2009) {\it  A Matlab differentiation matrix suite, ACM Trans. Math. Software 26:465-519}. 
\bibitem{2}{\sc Iserles, A.}(2006) {\it{A first course in the numerical analysis of differential equations}Cambridge Texts in Applied Mathematics, Cambridge University Press}.
\bibitem{2}{\sc LeVeque, R.J.}(2007) {\it Finite difference methods for ordinary and partial differential equations: Steady-state and time-dependent problems, SIAM, Philadelphia}.

\end{thebibliography}

\end{titlepage}
%%%%%%%%%%%%%%%%%% 16
\setcounter{section}{15}
\author{%
\tema{Matem\'atica Aplicada,}\vspace{2pt}\\
  Primitivo Acosta-Hum\'anez,\vspace{2pt} \\
       Departamento de Matematicas, \vspace{2pt}\\
       Universidad del Atl\'antico,\vspace{2pt}\\ 
        Intelectual.Co,\vspace{2pt}\\
        \hspace*{-2cm}\texttt{\scriptsize primi@intelectual.co }\vspace{10pt} \\
       Henock Venegas{2pt}\\
        Universidad del Atl\'antico,\vspace{2pt}\\
               \hspace*{-2cm}\texttt{\scriptsize henockv93@gmail.com }\vspace{20pt} \\
              %\hspace*{-2cm}\texttt{\scriptsize lhhurtado@uniquindio.edu.co}\\
         }
\pagecolor{white}
\pagestyle{eimat}
%\setcounter{section}{2}
\begin{titlepage}
\pagecolor{white}
%%%%%%%definiciones
\newcommand{\R}{\ensuremath{\mathbb{R}}}
%%%%%%%%%%
%\pagecolor{yellow}\'afterpage{\nopagecolor}
%\pagestyle{plain}
\BgThispage
\newgeometry{left=2cm,right=2cm,top=2cm,bottom=2cm}
\vspace*{-1.1cm}
\noindent
\def\titulo#1{\section{#1}}

\section{\bf\large\textcolor{white}{Algunas propiedades algebraicas de los grupos del tipo $(p,q)$ y grupos diedros }}
\vspace*{2cm}\par
\noindent

\begin{minipage}{0.5\linewidth}
\begin{minipage}{0.45\linewidth}
    \begin{flushright}
        \printauthor
    \end{flushright}
\end{minipage} \hspace{-3pt}
%
\begin{minipage}{0.02\linewidth}
   \color{ptctitle} \rule{1pt}{280pt}
\end{minipage} 
\end{minipage}
\hspace*{-4.5cm}
%
\begin{minipage}{0.85\linewidth}
\begin{minipage}{0.85\linewidth}
\footnotesize
\vspace{5pt}
    \begin{resumen}
 En este poster se presentar\'an ejemplos de c\'omo utilizar la teor\'{\i}a de Galois diferencial para construir expl\'{\i}citamente las soluciones de la ecuaci\'on estacionaria y unidimensional de Schr\"odinger con potenciales polinomiales. En particular se estudiar\'a el oscilador arm\'onico y los osciladores anarm\'onicos cu\'artico y s\'extico. Este poster es un avance de la tesis del segundo expositor.
    \end{resumen}
   % \hspace*{25pt}\palabras{Diferencias finitas; electromagnetismo; simulaci\'on; Matlab\circledR.}
\end{minipage}
\vspace*{5pt}\\
\footnotesize
%\begin{minipage}{0.85\linewidth}
%\vspace{5pt}
%    \begin{abstract} 
%  
%\lipsum[3-4]
%    \end{abstract}\vspace{5pt}
%     \hspace*{25pt}\keywords{one, two, three, four}
    
%\end{minipage}
\end{minipage}
\vspace{5pt}
\begin{thebibliography}{99}
\bibitem{PB}{\sc P.B. Acosta-Hum\'anez}, {\it  Galoisian Approach to Supersymmetric Quantum Mechanics: The integrability analysis of the Schr\"odinger equation by means of differential Galois theory}.  VDM Verlag Dr Mueller Publishing, Berlin, 2010.

\bibitem{PB2} {\sc P.B. Acosta-Hum\'anez, J.J. Morales-Ruiz, J.A. Weil} {\it Galoisian approach to integrability of Schr\"odinger equation}. Reports on Mathematical Physics 67 (3), 305-374.
\end{thebibliography}

\end{titlepage}
%%%%%%%%%%%%%%%%%%% 17
%\setcounter{section}{15}
\author{%
\tema{Teor\'{\i}a de Galois Diferencial,}\vspace{2pt}\\
  Primitivo Acosta-Hum\'anez,\vspace{2pt} \\
       Departamento de Matematicas, \vspace{2pt}\\
       Universidad del Atl\'antico,\vspace{2pt}\\ 
        Intelectual.Co,\vspace{2pt}\\
        \hspace*{-2cm}\texttt{\scriptsize primi@intelectual.co }\vspace{10pt} \\
      Jos\'e Paternina Amador,\vspace{2pt}\\
        Universidad del Atl\'antico,\vspace{2pt}\\
                 \hspace*{-2cm}\texttt{\scriptsize jpaterninac@uninorte.edu.co }\vspace{20pt} \\
              %\hspace*{-2cm}\texttt{\scriptsize lhhurtado@uniquindio.edu.co}\\
         }
\pagecolor{white}
\pagestyle{eimat}
%\setcounter{section}{2}
\begin{titlepage}
\pagecolor{white}
%%%%%%%definiciones
\newcommand{\R}{\ensuremath{\mathbb{R}}}
%%%%%%%%%%
%\pagecolor{yellow}\'afterpage{\nopagecolor}
%\pagestyle{plain}
\BgThispage
\newgeometry{left=2cm,right=2cm,top=2cm,bottom=2cm}
\vspace*{-1.1cm}
\noindent
\def\titulo#1{\section{#1}}

\section{\bf\large\textcolor{white}{An\'alisis Galoisiano de Ecuaciones de Schr\"odinger con potenciales polinomiales. }}
\vspace*{2cm}\par
\noindent

\begin{minipage}{0.5\linewidth}
\begin{minipage}{0.45\linewidth}
    \begin{flushright}
        \printauthor
    \end{flushright}
\end{minipage} \hspace{-3pt}
%
\begin{minipage}{0.02\linewidth}
   \color{ptctitle} \rule{1pt}{280pt}
\end{minipage} 
\end{minipage}
\hspace*{-4.5cm}
%
\begin{minipage}{0.85\linewidth}
\begin{minipage}{0.85\linewidth}
\footnotesize
\vspace{5pt}
    \begin{resumen}
 En este poster se presentan algunas propiedades de los grupos finitos no conmutativos generados por dos elementos, es decir, los grupos del tipo $(p,q)$. Entre las propiedades que se presentar\'an se encuentran su presentaci\'on y su representaci\'on en $GL(2,R)$ y en $GL(2,C)$. Tambi\'en se hace \'enfasis en los teoremas de isomorf\'ia y en la resolubidad de tales grupos.
    \end{resumen}
   % \hspace*{25pt}\palabras{Diferencias finitas; electromagnetismo; simulaci\'on; Matlab\circledR.}
\end{minipage}
\vspace*{5pt}\\
\footnotesize
%\begin{minipage}{0.85\linewidth}
%\vspace{5pt}
%    \begin{abstract} 
%  
%\lipsum[3-4]
%    \end{abstract}\vspace{5pt}
%     \hspace*{25pt}\keywords{one, two, three, four}
    
%\end{minipage}
\end{minipage}
\vspace{5pt}
\begin{thebibliography}{99}

\bibitem{acos} {\sc Acosta-Hum\'anez, P. B} (2003) "Grupos dihedros y del tipo $(p,q)$", tesis de pregrado. Universidad Sergio Arboleda.
\bibitem{acos}{\sc Acosta-Hum\'anez, P. B} (2003) "Teoremas de isomorf\'ia en grupos dihedros". Lecturas matem\'aticas, Vol. 24, p\'ags. 123-126.
\bibitem{acos}{\sc Charris Casta\~neda J.A, \sc Aldana G\'omez B., Acosta-Hum\'anez, P. B} (2013) "\'Algebra: Fundamentos, grupos, anillos, cuerpos y teor\'ia de Galois". Academia colombiana de ciencias exactas, f\'isicas y naturales. Colecci\'on Julio Carrizosa Valenzuela No.16. Bogot\'a D.C. 
\end{thebibliography}

\end{titlepage}
%%%%%%%%%%%%%%%%%% 18
%\setcounter{section}{15}
\author{%
\tema{Matem\'atica Aplicada,}\vspace{2pt}\\
  Theran S. L,\vspace{2pt} \\
  Racedo N. F,\vspace{2pt}
      Grupo GEOEL, Prog. F\'isica,\vspace{2pt}\
       Universidad del Atl\'antico. Colombia, \vspace{2pt}\\
              \hspace*{-2cm}\texttt{\scriptsize fran@mail.uniatlantico.edu.co }\vspace{10pt} \\
     Valbuena D. S,\vspace{2pt}\\
        Universidad del Atl\'antico,\vspace{2pt}\\
                              %\hspace*{-2cm}\texttt{\scriptsize lhhurtado@uniquindio.edu.co}\\
         }
\pagecolor{white}
\pagestyle{eimat}
%\setcounter{section}{2}
\begin{titlepage}
\pagecolor{white}
%%%%%%%definiciones
\newcommand{\R}{\ensuremath{\mathbb{R}}}
%%%%%%%%%%
%\pagecolor{yellow}\'afterpage{\nopagecolor}
%\pagestyle{plain}
\BgThispage
\newgeometry{left=2cm,right=2cm,top=2cm,bottom=2cm}
\vspace*{-1.1cm}
\noindent
\def\titulo#1{\section{#1}}

\section{\bf\large\textcolor{white}{Desarrollo de una gui en Matlab\circledR\ del m\'etodo FDTD para la simulaci\'on dell fen\'omeno de difracci\'on}}
\vspace*{2cm}\par
\noindent

\begin{minipage}{0.5\linewidth}
\begin{minipage}{0.45\linewidth}
    \begin{flushright}
        \printauthor
    \end{flushright}
\end{minipage} \hspace{-3pt}
%
\begin{minipage}{0.02\linewidth}
   \color{ptctitle} \rule{1pt}{140pt}
\end{minipage} 
\end{minipage}
\hspace*{-4.5cm}
%
\begin{minipage}{0.85\linewidth}
\begin{minipage}{0.85\linewidth}
\footnotesize
\vspace{5pt}
    \begin{resumen}
 El m\'etodo de diferencias finitas en el dominio del tiempo (FDTD) ha demostrado ser una herramienta \'util para el an\'alisis de los fen\'omenos electromagn\'eticos. En este trabajo se presenta una aplicaci\'on grafica basada en la GUIDE de MATLAB\circledR, en la cual se muestra la implementaci\'on del algoritmo de diferencias finitas en el dominio del tiempo, para la simulaci\'on del fen\'omeno de difracci\'on por una y dos rendijas. Se introducir\'an unas condiciones de frontera absorbentes (ABC) de tipo Capas perfectamente acopladas (PML), ya que este es un problema de evoluci\'on temporal con dominios no acotados.

    \end{resumen}
   % \hspace*{25pt}\palabras{Diferencias finitas; electromagnetismo; simulaci\'on; Matlab\circledR.}
\end{minipage}
\vspace*{5pt}\\
\footnotesize
%\begin{minipage}{0.85\linewidth}
%\vspace{5pt}
%    \begin{abstract} 
%  
%\lipsum[3-4]
%    \end{abstract}\vspace{5pt}
%     \hspace*{25pt}\keywords{one, two, three, four}
    
%\end{minipage}
\end{minipage}
\vspace{5pt}
\begin{thebibliography}{99}

\bibitem{1}{\sc Larry Theran, Ren\'e Alvarez, Sonia Valbuena and Francisco Racedo}(2014) {\it  Estudio Num\'erico De La Propagaci\'on De Ondas Electromagn\'eticas 2-D Por FDTD, Revista de Matem\'atica MATUA, ISSN: 2389-7422  }. 
\bibitem{2}{\sc Dennis M. Sullivan.}(June 2013, Wiley-IEEE Press) {\it{Electromagnetic Simulation Using the FDTD Method, 2nd Edition}Cambridge Texts in Applied Mathematics, ISBN: 978-1-118-45939-3}.
\bibitem{2}{\sc  Allen Taflove, Susan C. Hagness.}(May 31, 2005) {\it Computational Electrodynamics: The Finite-Difference Time-Domain Method, Third Edition, ISBN-13: 978-1580538329 ISBN-10: 1580538320}.
\end{thebibliography}

\end{titlepage}
%%%%%%%%%%%%%%% 20
%%%%%%%%%%%%%%%%%%%%% 
\setcounter{section}{19}
\author{%
\tema{Estad\'istica,}\vspace{10pt}\\
CARLOS V\'{E}LEZ MIRANDA,\vspace{2pt} \\
Grupo GIMA.\vspace{2pt} \\
 Universidad del Atl\'antico,\vspace{2pt} \\
    \hspace*{-2cm}\texttt{\scriptsize carlosvelezmiranda@hotmail.com }\vspace{10pt} \\
         }
\pagecolor{white}
\pagestyle{eimat}
%\setcounter{section}{2}
\begin{titlepage}
\pagecolor{white}
%%%%%%%definiciones
\newcommand{\R}{\ensuremath{\mathbb{R}}}
%%%%%%%%%%
%\pagecolor{yellow}\'afterpage{\nopagecolor}
%\pagestyle{plain}
\BgThispage
\newgeometry{left=2cm,right=2cm,top=2cm,bottom=2cm}
\vspace*{-1.1cm}
\noindent
\def\titulo#1{\section{#1}}

\section{\bf\large\textcolor{white}{Soporte de Dise\~no A-\'{O}ptimo en Modelos de regresi\'{o}n Polin\'{o}mica}}
\vspace*{2cm}\par
\noindent

\begin{minipage}{0.5\linewidth}
\begin{minipage}{0.45\linewidth}
    \begin{flushright}
        \printauthor
    \end{flushright}
\end{minipage} \hspace{-3pt}
%
\begin{minipage}{0.02\linewidth}
   \color{ptctitle} \rule{1pt}{140pt}
\end{minipage} 
\end{minipage}
\hspace*{-4.5cm}
%
\begin{minipage}{0.85\linewidth}
\begin{minipage}{0.85\linewidth}
\footnotesize
\vspace{5pt}
    \begin{resumen}
 Se ilustra la b\'{u}squeda de los puntos de soporte del dise\~no A-\'{o}ptimo para un modelo polin\'{o}mico de grado $m$; en este caso el soporte tiene no m\'{a}s de $m+1$ puntos, correspondientes a las ra\'{i}ces del  polinomio $(1-x^{2})P'_{m}(x)$; donde $P_{m}(x)$ es el polinomio de Legendre de grado $m$. El algoritmo usado en la construcci\'{o}n del dise\~no A-\'{o}ptimo bajo el supuesto de homocedasticidad, fue desarrollado con el programa estad\'{i}stico R-project, se presenta un ejemplo de implementaci\'{o}n del mismo.
    \end{resumen}
   % \hspace*{25pt}\palabras{Diferencias finitas; electromagnetismo; simulaci\'on; Matlab\circledR.}
\end{minipage}
\vspace*{5pt}\\
\footnotesize
%\begin{minipage}{0.85\linewidth}
%\vspace{5pt}
%    \begin{abstract} 
%  
%\lipsum[3-4]
%    \end{abstract}\vspace{5pt}
%     \hspace*{25pt}\keywords{one, two, three, four}
    
%\end{minipage}
\end{minipage}
\vspace{5pt}
\begin{thebibliography}{99}

\bibitem{Chang}{\sc Chang, F.C. y Yeh, Y.R.} (1998), ``Exact A-Optimal Designs For Quadratic Regression''.  Statistica Sinica, 8, 527 - 533.

\bibitem{Erma} {\sc Ermakov, S. M. y Zhiglijavsky, A. A. } (1987), ``The Mathematical Theory of Optimum Experiments''. \emph{Nauka}. Moscow (In Russian).

\bibitem{karlin} {\sc Karlin, S. y Studden, W.J.} (1996), "Optimal experimental designs''. \emph{Ann. Math. Statist, 37}, 783 - 815
\bibitem{maly} {\sc Malyutov, M.B. y Fedorov, B.B.} (1969), {\it "On the designs for certain weighted polynomial regression minimizing the average variance''}. \emph{Prepint No. 8, Editorial Universidad de Moscu}, 734 - 738
\bibitem{men} {\sc Mendehall, W. y Sincich, T.} (2012), "A Second Course in Statistics: Regression Analysis''. \emph{University of Florida}

\bibitem{puk1} {\sc Pukelsheim, F. y Studden, W.J.} (1993), "E-optimal dessgins for polynomial regression''. \emph{Ann. Statist.,21,} 402 - 415
\bibitem{rod} {\sc Rodriguez, C. y Ortiz, I.} (2000), "Dise\~no Optimo Para Modelos de regression''. \emph{Universidad de Almer\'{i}a,} Espa\~na
\bibitem{su} {\sc Su, Y.C.} (2005), "A-optimal designs for weighted polynomial regression''. \emph{Thesys . Department of applied mathematics, National Sun Yat-sen University, Kaohsiung} Taiwan
\end{thebibliography}

\end{titlepage}
%%%%%%%%%%%%%%%%%%%%% 22
\setcounter{section}{21}
\author{%
\tema{Matem\'atica Aplicada,}\vspace{10pt}\\
Carlos Jim\'enez,\vspace{2pt} \\
Grupo GIMA.\vspace{2pt} \\
 Universidad de la Guajira,\vspace{2pt} \\
  Riohacha, Colombia.\vspace{2pt} \\
  \hspace*{-2cm}\texttt{\scriptsize carlosj114@gmail.com }\vspace{10pt} \\
Susana Salinas de Romero,\vspace{2pt} \\
         CIMA. Universidad del Zulia,\vspace{2pt}\\
          Centro de investigaci\'on en matem\'atica aplicada.,\vspace{2pt}\\
 Maracaibo-Venezuela, \vspace{10pt}\\
 Cesar. Torres,\vspace{2pt}\\
 LOI. Laboratorio de \'optica e inform\'atica.,\vspace{2pt}\\
  Universidad Popular del
Cesar . ,\vspace{2pt}\\Valledupar. Colombia,\vspace{2pt}\\                               
         }
\pagecolor{white}
\pagestyle{eimat}
%\setcounter{section}{2}
\begin{titlepage}
\pagecolor{white}
%%%%%%%definiciones
\newcommand{\R}{\ensuremath{\mathbb{R}}}
%%%%%%%%%%
%\pagecolor{yellow}\'afterpage{\nopagecolor}
%\pagestyle{plain}
\BgThispage
\newgeometry{left=2cm,right=2cm,top=2cm,bottom=2cm}
\vspace*{-1.1cm}
\noindent
\def\titulo#1{\section{#1}}

\section{\bf\large\textcolor{white}{Desarrollo hist\'orico de la transformada fraccional de Fourier.}}
\vspace*{2cm}\par
\noindent

\begin{minipage}{0.5\linewidth}
\begin{minipage}{0.45\linewidth}
    \begin{flushright}
        \printauthor
    \end{flushright}
\end{minipage} \hspace{-3pt}
%
\begin{minipage}{0.02\linewidth}
   \color{ptctitle} \rule{1pt}{140pt}
\end{minipage} 
\end{minipage}
\hspace*{-4.5cm}
%
\begin{minipage}{0.85\linewidth}
\begin{minipage}{0.85\linewidth}
\footnotesize
\vspace{5pt}
    \begin{resumen}
 La transformada fraccional de Fourier (FrFT), es una generalizaci\'on de la
transformada cl\'asica de Fourier [1,2], la cual fue introducida en la
literatura matem\'atica por Wiener 1929[2], posteriormente V\'ictor Namias en
1980 [1], desarrolla las diferentes propiedades para dicha transformada con
algunas aplicaciones a la mec\'anica cu\'antica. Fueron Lohmann y Mendlovic
quienes en 1993 [3 -- 5], introdugeron distitas aplicaciones en \'optica. En
este trabajo se presenta el desarrollo hist\'orico de dicha transformada,
desde sus inicios en 1929 hasta la fecha; se presentan adem\'as algunas
aplicaciones en la \'optica y procesamiento de se\~nales, haciendo uso de la
plataforma de MaTlab.

Palabras claves: transformada fraccional de Fourier, mec\'anica cu\'antica,
procesamiento de se\~nales, sistemas \'opticos.
    \end{resumen}
   % \hspace*{25pt}\palabras{Diferencias finitas; electromagnetismo; simulaci\'on; Matlab\circledR.}
\end{minipage}
\vspace*{5pt}\\
\footnotesize
%\begin{minipage}{0.85\linewidth}
%\vspace{5pt}
%    \begin{abstract} 
%  
%\lipsum[3-4]
%    \end{abstract}\vspace{5pt}
%     \hspace*{25pt}\keywords{one, two, three, four}
    
%\end{minipage}
\end{minipage}
\vspace{5pt}
\begin{thebibliography}{99}

\bibitem{1} \qquad Victor Namias. The Fractional Order Fourier. Transform
and its Applications to Quantum Mechanics. J. Inst. Maths. Applics. No. 25.
1980. pp 241-265

\bibitem{2} \qquad Haldun M.Ozaktas. Zeev Zalevsky. M. Alper Kutay. The
Fractional Fourier Transform with Applications in Optics and Signal
Processing. John Wiley \& Sons. LTD. New. York. 2001.

\bibitem{3} \qquad Joseph W. Goodman. Introduction to Fourier Optics.
McGraw-Hill Companies. Inc.New York. 1.996.

\bibitem{4} \qquad Bladimir Vega. Jaider Pe\~na. y Cesar Torres An\'alisis de
Sistemas \'opticos Multielementos utilizando el operador Transformada de
Fourier de Orden Fraccional. Revista Colombiana de F\'isica. Vol.38 (1). 2006.
pp. 61-65.

\bibitem{5} \qquad William Lohmann. Imaginer Rotation and The Fractional
Fourier Transform Jornal.Optics. Soc.Am. No. A10. 1993. pp. 2181-2186.

L\'inea de Investigaci\'on: Matem\'aticas aplicadas \ \ \ \ \ \ \ \ \ \ \
Modalidad: Poster
\end{thebibliography}

\end{titlepage}
%%%%%%%%%%%%% 23
\setcounter{section}{21}
\author{%
\tema{Matem\'{a}tica Vigesimal,}\vspace{10pt}\\
Maria Angelica Serje Arias,\vspace{2pt} \\
Grupo GIMA.\vspace{2pt} \\
  Universidad del Atlantico,\vspace{2pt} \\
   \hspace*{-2cm}\texttt{\scriptsize mariangel3123@hotmail.com }\vspace{10pt} \\
         }
\pagecolor{white}
\pagestyle{eimat}
%\setcounter{section}{2}
\begin{titlepage}
\pagecolor{white}
%%%%%%%definiciones
\newcommand{\R}{\ensuremath{\mathbb{R}}}
%%%%%%%%%%
%\pagecolor{yellow}\'afterpage{\nopagecolor}
%\pagestyle{plain}
\BgThispage
\newgeometry{left=2cm,right=2cm,top=2cm,bottom=2cm}
\vspace*{-1.1cm}
\noindent
\def\titulo#1{\section{#1}}

\section{\bf\large\textcolor{white}{Matem\'{a}tica Vigesimal: Una mirada desde la filosof\'{\i}a del numero maya.}}
\vspace*{2cm}\par
\noindent

\begin{minipage}{0.5\linewidth}
\begin{minipage}{0.45\linewidth}
    \begin{flushright}
        \printauthor
    \end{flushright}
\end{minipage} \hspace{-3pt}
%
\begin{minipage}{0.02\linewidth}
   \color{ptctitle} \rule{1pt}{190pt}
\end{minipage} 
\end{minipage}
\hspace*{-4.5cm}
%
\begin{minipage}{0.85\linewidth}
\begin{minipage}{0.85\linewidth}
\footnotesize
\vspace{5pt}
    \begin{resumen}
 En todo el territorio maya que abarca toda la parte sur de M\'{e}xico y Centro Am\'{e}rica, la pronunciaci\'{o}n de los vocablos num\'{e}ricos manifiestan muy poca variaci\'{o}n.
\\La filosof\'{\i}a de los n\'{u}meros mayas, su cosmovisi\'{o}n del numero para mi es algo que absorbe a las personas entre mas consultan, leen e investigan sobre esto, la primera genialidad es la invenci\'{o}n del cero que empez\'{o} a ser utilizado, mucho antes del inicio de la fecha cristiana. Este dato  le da renombre al sistema maya en el mundo cient\'{\i}fico.  Adelant\'{a}ndose casi mil a\~{n}os al cero indoarabigo que se utiliza en las escuelas, otra genialidad es que tan solo utiliza tres s\'{\i}mbolos, el s\'{\i}mbolo del cero, el de la unidad y el cinco. Con estos incre\'{\i}bles s\'{\i}mbolos se construy\'{o} la milenaria cultura maya, que ha asombrado al mundo.  El sistema vigesimal maya resulta entonces, eficiente y simple.\\
Los n\'{u}meros concretos son:  el caracolillo, la flor, la semilla ovalada de c\'{a}scara dura (semilla de zapote, de durazno), \'{e}stos se utilizan como Cero.  Como unidad se utilizan piedrecillas, botones, frijoles y similares.  Como cinco se utilizan, palillos y paletitas. Los dedos de la mano son los otros instrumentos num\'{e}ricos, porque los 20 dedos de la persona, fueron tomados en cuenta para la construcci\'{o}n de este sistema. Por esa raz\'{o}n el numero veinte se llama \emph{junwinaq,  junwinq}  que significa \emph{una persona }haciendo alusi\'{o}n a los dedos de una persona completa.La pr\'{a}ctica de la matem\'{a}tica maya es una experiencia formidable, porque  al entrar contacto con los tres s\'{\i}mbolos concretos antes mencionados, \'{e}stos  te hacen llevar al infinito, con la facilidad de entender los n\'{u}meros tallados en las estelas. Los n\'{u}meros no solo son representaciones, sino que hacen parte de la naturaleza.
Mayas.
    \end{resumen}
   % \hspace*{25pt}\palabras{Diferencias finitas; electromagnetismo; simulaci\'on; Matlab\circledR.}
\end{minipage}
\vspace*{5pt}\\
\footnotesize
%\begin{minipage}{0.85\linewidth}
%\vspace{5pt}
%    \begin{abstract} 
%  
%\lipsum[3-4]
%    \end{abstract}\vspace{5pt}
%     \hspace*{25pt}\keywords{one, two, three, four}
    
%\end{minipage}
\end{minipage}
\vspace{5pt}
\begin{thebibliography}{99}
\bibitem{Ad}{\sc Jose Mucia Batz} (2010)  Matematica Vigesimal Maya.
\bibitem{bav1} {\sc Jose Mucia Batz} NIK Filosofia de los numeros
\end{thebibliography}

\end{titlepage}